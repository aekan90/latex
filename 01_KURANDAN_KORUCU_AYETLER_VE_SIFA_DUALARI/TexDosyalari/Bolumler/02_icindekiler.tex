\documentclass[12pt,a4paper]{article}
\usepackage[utf8]{inputenc}
\usepackage[T1]{fontenc}
\usepackage{geometry}

\geometry{margin=2.5cm}
\setlength{\parindent}{0pt}
\setlength{\parskip}{1em}

\begin{document}
\section{İÇİNDEKİLER}
\subsection{1. ÖNSÖZ}
\subsubsection{1.1. Kitabın Amacı}
\subsubsection{1.2. Kullanım Rehberi}
\subsection{2. GİRİŞ}
\subsubsection{2.1. Önemli Notlar}
\subsubsection{2.2. Açıklama}
\subsection{3. PRATİK KULLANIM REHBERİ}
\subsubsection{3.1. Günlük Okunacak Dualar}
\subsubsection{3.2. Özel Durumlar İçin Dualar ve Uygulamalar}
\subsubsection{3.3. Muska Kullanımı}
\paragraph{3.3.1. Koruma Muskası}
\paragraph{3.3.2. Şifa Muskası}
\paragraph{3.3.3. Huzur Muskası}
\paragraph{3.3.4. Bereket Muskası}
\paragraph{3.3.5. Aile Saadeti Muskası}
\paragraph{3.3.6. Rızık Muskası}
\paragraph{3.3.7. Nazar ve Haset Muskası}
\paragraph{3.3.8. Evlilik Muskası}
\subsubsection{3.4. Özel Günler ve Tekrarlar}
\subsubsection{3.5. Zor Durumlar İçin Özel Uygulamalar}
\subsubsection{3.6. Hanefi-Maturidi Mezhebine Göre Özel Uygulamalar}
\subsubsection{3.7. Günlük Zikir Programı}
\subsection{4. TELAFFUZ VE SURELERİN ÖĞRENİMİ}
\subsubsection{4.1. Temel Harflerin Telaffuzu}
\subsubsection{4.2. Surelerin Okunuşu}
\subsection{5. TEMEL SURELER}
\subsubsection{5.1. Fatiha Suresi (1:1–7)}
\subsubsection{5.2. Ayet-el-Kursi (Bakara 255)}
\subsubsection{5.3. İhlas Suresi (112:1-4)}
\subsubsection{5.4. Felak Suresi (113:1-5)}
\subsubsection{5.5. Nas Suresi (114:1-6)}
\subsubsection{5.6. İnşirah Suresi (94:1-8)}
\subsection{6. KORUMA VE ŞİFA AYETLERİ}
\subsubsection{6.1. Bakara 285-286}
\subsubsection{6.2. İsra 82}
\subsubsection{6.3. Şuara 80-81}
\subsubsection{6.4. Müminun 98-99}
\subsubsection{6.5. Furkan 74-75}
\subsubsection{6.6. Rad 28}
\subsubsection{6.7. İnşirah 5-6}
\subsubsection{6.8. Al-İmran 103}
\subsubsection{6.9. İbrahim 7}
\subsubsection{6.10. Taha 124-126}
\subsubsection{6.11. Yasin 82 ve Rahman 33}
\subsubsection{6.12. Fussilet 44-46}
\subsection{7. GÜNLÜK ZİKİRLER}
\subsubsection{7.1. Sübhanallahi ve bihamdihi}
\subsubsection{7.2. La ilahe illallah}
\subsubsection{7.3. Allahümme salli ala Muhammed}
\subsubsection{7.4. Estağfirullah}
\subsubsection{7.5. La havle vela kuvvete illa billah}
\subsection{8. ÖZEL DURUMLAR İÇİN ZİKİRLER}
\subsubsection{8.1. Euzü billahi mineşşeytanirracim}
\subsubsection{8.2. Hasbünallahü ve ni'mel vekil}
\subsubsection{8.3. La ilahe illa ente sübhaneke inni küntü minezzalimin}
\subsubsection{8.4. Euzü biizzetillahi ve kudretihi min şerri ma ecidü ve uhaziru}
\subsection{9. KAPANIŞ}
\subsection{10. KAYNAKÇA}
\subsubsection{10.1. Kur'an-ı Kerim ve Mealleri}
\subsubsection{10.2. Hadis Kaynakları}
\subsubsection{10.3. Fıkıh Kaynakları}
\subsubsection{10.4. Tefsir Kaynakları}
\subsubsection{10.5. Tasavvuf Kaynakları}
\subsubsection{10.6. Modern Kaynaklar}
\subsubsection{10.7. İnternet Kaynakları}
---

\end{document}
