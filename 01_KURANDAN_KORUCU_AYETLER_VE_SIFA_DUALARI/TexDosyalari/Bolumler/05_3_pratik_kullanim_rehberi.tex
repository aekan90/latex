\documentclass[12pt,a4paper]{article}
\usepackage[utf8]{inputenc}
\usepackage[T1]{fontenc}
\usepackage{geometry}

\geometry{margin=2.5cm}
\setlength{\parindent}{0pt}
\setlength{\parskip}{1em}

\begin{document}
\section{3. PRATİK KULLANIM REHBERİ}
\subsection{3.1. Günlük Okunacak Dualar}
\textbf{Rivayet:} Bu program, Hz. Peygamber'in (s.a.v.) günlük uygulamalarına dayanmaktadır. Buhari'de geçen hadislere göre, Hz. Peygamber (s.a.v.) belirli vakitlerde özel sureleri okurdu. (Buhari, Daavat, 64) İmam Gazali, "İhya-u Ulumid-Din" adlı eserinde bu uygulamaların faziletlerini detaylı olarak anlatmıştır.
\begin{itemize}
\item Sabah: Ayet-el-Kursi (1 kez)
\item Öğle: İhlas Suresi (3 kez)
\item İkindi: Felak ve Nas Sureleri (1'er kez)
\item Akşam: Bakara 285-286 (1 kez)
\item Yatsı: Tüm temel sureler (1'er kez)
\end{itemize}
\textbf{Not:} Ayet-el-Kursi, İhlas, Felak ve Nas Surelerinin tam metinleri Bölüm 5'te bulunmaktadır.
\textbf{Bakara 285-286:}
آمَنَ الرَّسُولُ بِمَا أُنزِلَ إِلَيْهِ مِن رَّبِّهِ وَالْمُؤْمِنُونَ ۚ كُلٌّ آمَنَ بِاللَّهِ وَمَلَائِكَتِهِ وَكُتُبِهِ وَرُسُلِهِ لَا نُفَرِّقُ بَيْنَ أَحَدٍ مِّن رُّسُلِهِ ۚ وَقَالُوا سَمِعْنَا وَأَطَعْنَا ۖ غُفْرَانَكَ رَبَّنَا وَإِلَيْكَ الْمَصِيرُ
*Âmener rasûlü bimâ ünzile ileyhi min rabbihî vel mü'minûn. Küllün âmene billâhi ve melâiketihî ve kütübihî ve rusülih. Lâ nüferriku beyne ehadin min rusülih. Ve kâlû semi'nâ ve eta'nâ ğufrâneke rabbenâ ve ileykel masîr*
(Peygamber, Rabbinden kendisine indirilene iman etti, müminler de (iman ettiler). Her biri Allah'a, meleklerine, kitaplarına, peygamberlerine iman ettiler. "Allah'ın peygamberlerinden hiçbiri arasında ayırım yapmayız. İşittik, itaat ettik. Ey Rabbimiz, affına sığındık! Dönüş sanadır" dediler)
لَا يُكَلِّفُ اللَّهُ نَفْسًا إِلَّا وُسْعَهَا ۚ لَهَا مَا كَسَبَتْ وَعَلَيْهَا مَا اكْتَسَبَتْ ۗ رَبَّنَا لَا تُؤَاخِذْنَا إِن نَّسِينَا أَوْ أَخْطَأْنَا ۚ رَبَّنَا وَلَا تَحْمِلْ عَلَيْنَا إِصْرًا كَمَا حَمَلْتَهُ عَلَى الَّذِينَ مِن قَبْلِنَا ۚ رَبَّنَا وَلَا تُحَمِّلْنَا مَا لَا طَاقَةَ لَنَا بِهِ ۖ وَاعْفُ عَنَّا وَاغْفِرْ لَنَا وَارْحَمْنَا ۚ أَنتَ مَوْلَانَا فَانصُرْنَا عَلَى الْقَوْمِ الْكَافِرِينَ
*Lâ yükellifullâhü nefsen illâ vüs'ahâ. Lehâ mâ kesebet ve aleyhâ mektesebet. Rabbenâ lâ tüâhiznâ in nesînâ ev ahta'nâ. Rabbenâ ve lâ tahmil aleynâ isran kemâ hameltehû alellezîne min kablinâ. Rabbenâ ve lâ tühammilnâ mâ lâ tâkate lenâ bih. Va'fü annâ, vağfir lenâ, verhamnâ. Ente mevlânâ fensurnâ alel kavmil kâfirîn*
(Allah bir kimseyi ancak gücünün yettiği şeyle yükümlü kılar. Onun kazandığı iyilik kendi yararına, kötülük de kendi zararınadır. "Ey Rabbimiz! Unutur ya da yanılırsak bizi sorumlu tutma. Ey Rabbimiz! Bize, bizden öncekilere yüklediğin gibi ağır yük yükleme. Ey Rabbimiz! Bize gücümüzün yetmediği şeyleri yükleme. Bizi affet, bizi bağışla, bize acı! Sen bizim Mevlâmızsın. Kâfirler topluluğuna karşı bize yardım et!")
\subsection{3.2. Özel Durumlar İçin Dualar ve Uygulamalar}
\textbf{Bölüm Özeti:}
Bu bölümde, özel durumlar için okunması gereken dualar ve uygulamalar ele alınmaktadır. Her bir durum için Hz. Peygamber'in (s.a.v.) tavsiye ettiği sureler ve ayetler, rivayetlerle birlikte sunulmuştur. Ayrıca, bu duaların nasıl ve ne zaman okunması gerektiği konusunda pratik bilgiler verilmiştir.
*Rivayet:* Bu zikirler, Hz. Peygamber'in (s.a.v.) özel durumlarda okuduğu dualardır. Tirmizi'de geçen hadislere göre, Hz. Peygamber (s.a.v.) vesvese, korku ve sıkıntı durumlarında özel zikirleri okurdu. (Tirmizi, Daavat, 101) İmam Rabbani, "Mektubat" adlı eserinde bu uygulamaların önemini vurgulamıştır.
\subsubsection{Sıkıntılı Anlarda}
\textbf{İnşirah Suresi} (7 kez)
Bu sure tam metin olarak Bölüm 5.6'da yer almaktadır. Lütfen oradan okuyunuz.
\subsubsection{Hastalık Durumunda}
\textbf{Şuara 80-81} (41 kez)
وَإِذَا مَرِضْتُ فَهُوَ يَشْفِينِ
*Ve izâ meridtü fe hüve yeşfîn*
(Hastalandığım zaman bana şifa veren O'dur)
وَالَّذِي يُمِيتُنِي ثُمَّ يُحْيِينِ
*Vellezîne yümîtünî sümme yuhyîn*
(O, beni öldürecek ve sonra diriltecek olandır)
\subsubsection{Ruhsal Huzur İçin}
\textbf{Rad 28} (33 kez)
الَّذِينَ آمَنُوا وَتَطْمَئِنُّ قُلُوبُهُم بِذِكْرِ اللَّهِ ۗ أَلَا بِذِكْرِ اللَّهِ تَطْمَئِنُّ الْقُلُوبُ
*Ellezîne âmenû ve tatmainnü kulûbühüm bi zikrillâh. E lâ bi zikrillâhi tatmainnül kulûb*
(Onlar, iman edenler ve kalpleri Allah'ı anmakla huzura kavuşanlardır. Dikkat edin, kalpler ancak Allah'ı anmakla huzur bulur)
\subsubsection{Maddi Sıkıntılar İçin}
\textbf{İbrahim 7} (21 kez)
وَإِذْ تَأَذَّنَ رَبُّكُمْ لَئِن شَكَرْتُمْ لَأَزِيدَنَّكُمْ ۖ وَلَئِن كَفَرْتُمْ إِنَّ عَذَابِي لَشَدِيدٌ
*Ve iz teezzene rabbüküm le in şekertüm le ezîdenneküm, ve le in kefertüm inne azâbî le şedîd*
(Rabbiniz şöyle duyurmuştu: "Andolsun, eğer şükrederseniz elbette size nimetimi artırırım. Eğer nankörlük ederseniz, hiç şüphesiz azabım çok şiddetlidir")
\subsubsection{Vesvese ve Şeytan'dan Korunma}
\textbf{Euzü billahi mineşşeytanirracim} (7 kez)
أَعُوذُ بِاللهِ مِنَ الشَّيْطَانِ الرَّجِيمِ
*Euzü billahi mineşşeytanirracim*
(Kovulmuş şeytandan Allah'a sığınırım)
\subsubsection{Sıkıntı ve Zorluk Anında}
\textbf{Hasbünallahü ve ni'mel vekil} (33 kez)
حَسْبُنَا اللَّهُ وَنِعْمَ الْوَكِيلُ
*Hasbünallahü ve ni'mel vekil*
(Allah bize yeter. O ne güzel vekildir)
\subsubsection{Tövbe ve İstiğfar}
\textbf{La ilahe illa ente sübhaneke inni küntü minezzalimin} (70 kez)
لَا إِلَهَ إِلَّا أَنْتَ سُبْحَانَكَ إِنِّي كُنْتُ مِنَ الظَّالِمِينَ
*La ilahe illa ente sübhaneke inni küntü minezzalimin*
(Senden başka ilah yoktur. Seni tenzih ederim. Gerçekten ben zalimlerden oldum)
\subsubsection{Ağrı ve Sızı İçin}
\textbf{Euzü biizzetillahi ve kudretihi min şerri ma ecidü ve uhaziru} (21 kez)
أَعُوذُ بِعِزَّةِ اللَّهِ وَقُدْرَتِهِ مِنْ شَرِّ مَا أَجِدُ وَأُحَاذِرُ
*Euzü biizzetillahi ve kudretihi min şerri ma ecidü ve uhaziru*
(Hissettiğim ve korktuğum şeylerin şerrinden Allah'ın izzet ve kudretine sığınırım)
\subsection{3.3. Muska Kullanımı}
\textbf{Bölüm Özeti:}
Bu bölümde, muska kullanımı ile ilgili temel prensipler ve uygulamalar ele alınmaktadır. Hz. Peygamber'in (s.a.v.) ve sahabenin muska kullanımına dair uygulamaları, fıkhi hükümler ve pratik bilgiler sunulmuştur. Ayrıca, muska hazırlama ve kullanma şartları, dikkat edilmesi gereken hususlar ve uyarılar detaylı olarak açıklanmıştır.
\textbf{ÖNEMLİ NOT:} Bu bölümde bahsedilen tüm muskaların Arapça metinleri ve sure-ayet bilgileri, kitabın son sayfasında "Muska Metinleri" başlığı altında toplu olarak verilmiştir. Muska hazırlamak isteyenler için kolay kullanım sağlaması amacıyla derlenmiştir.
\textbf{Genel Rivayet:} Muska kullanımı konusunda İmam Serahsi, "el-Mebsut" adlı eserinde detaylı bilgiler vermiştir. Hz. Peygamber'in (s.a.v.) bazı sahabeye muska yazdığı ve bunların şifa vesilesi olduğu hadislerle sabittir. (Buhari, Tıb, 19)
\textbf{Temel Muska Çeşitleri:}
1. \textbf{Koruma Muskası}
\textbf{Rivayet:} Hz. Aişe'den (r.a.) rivayet edildiğine göre, Hz. Peygamber (s.a.v.) her gece yatağına girdiğinde bu sureleri okur ve ellerine üfleyerek vücudunu mesh ederdi. (Buhari, Fedailü'l-Kur'an, 14)
\textbf{İçerik:}
\begin{itemize}
\item Ayet-el-Kursi (Bkz: Bölüm 5.2)
\item İhlas Suresi (Bkz: Bölüm 5.3)
\item Felak ve Nas Sureleri (Bkz: Bölüm 5.4 ve 5.5)
\item Bakara Suresi'nin son iki ayeti (Bkz: Bölüm 6.1)
\end{itemize}
2. \textbf{Şifa Muskası}
\textbf{Rivayet:} Hz. Peygamber'in (s.a.v.) hastalık durumunda bu ayetleri okuduğu rivayet edilmiştir. (Müslim, Selam, 40)
\textbf{İçerik:}
\begin{itemize}
\item Şuara 80-81 (Bkz: Bölüm 6.3)
\item İsra 82:
\end{itemize}
وَنُنَزِّلُ مِنَ الْقُرْآنِ مَا هُوَ شِفَاءٌ وَرَحْمَةٌ لِّلْمُؤْمِنِينَ
*Ve nünezzilü minel kur'âni mâ hüve şifâün ve rahmetün lil mü'minîn*
(Biz Kur'an'dan öyle bir şey indiriyoruz ki o, müminler için şifa ve rahmettir)
\begin{itemize}
\item Yunus 57:
\end{itemize}
يَا أَيُّهَا النَّاسُ قَدْ جَاءَتْكُم مَّوْعِظَةٌ مِّن رَّبِّكُمْ وَشِفَاءٌ لِّمَا فِي الصُّدُورِ
*Yâ eyyühen nâsü kad câetküm mev'izatün min rabbiküm ve şifâün limâ fis sudûr*
(Ey insanlar! Size Rabbinizden bir öğüt, kalplerdekine bir şifa gelmiştir)
3. \textbf{Huzur Muskası}
\textbf{Rivayet:} Abdullah ibn Mesud'dan (r.a.) rivayet edildiğine göre, kalp huzuru için bu ayetlerin okunması tavsiye edilmiştir. (Tirmizi, Daavat, 82)
\textbf{İçerik:}
\begin{itemize}
\item Rad 28 (Bkz: Bölüm 6.6)
\item İnşirah 5-6 (Bkz: Bölüm 5.6)
\item Zümer 53:
\end{itemize}
قُلْ يَا عِبَادِيَ الَّذِينَ أَسْرَفُوا عَلَىٰ أَنفُسِهِمْ لَا تَقْنَطُوا مِن رَّحْمَةِ اللَّهِ
*Kul yâ ibâdiyellezîne esrefû alâ enfüsihim lâ taknetû min rahmetillâh*
(De ki: Ey kendilerinin aleyhine aşırı giden kullarım! Allah'ın rahmetinden ümidinizi kesmeyin)
4. \textbf{Bereket Muskası}
\textbf{Rivayet:} Hz. Peygamber'den (s.a.v.) rivayet edildiğine göre, bu ayetleri okuyana rızkında bereket verilir. (Tirmizi, Fedailü'l-Kur'an, 12)
\textbf{İçerik:}
\begin{itemize}
\item İbrahim 7 (Bkz: Bölüm 6.9)
\item Talak 2-3:
\end{itemize}
وَمَن يَتَّقِ اللَّهَ يَجْعَل لَّهُ مَخْرَجًا وَيَرْزُقْهُ مِنْ حَيْثُ لَا يَحْتَسِبُ
*Ve men yettekıllâhe yec'al lehû mahracen ve yerzukhu min haysu lâ yahtesib*
(Kim Allah'tan korkarsa, Allah ona bir çıkış yolu gösterir ve onu beklemediği yerden rızıklandırır)
5. \textbf{Aile Saadeti Muskası}
\textbf{Rivayet:} Hz. Peygamber (s.a.v.), evlenen çiftlere bu ayetleri tavsiye etmiştir. (Ebu Davud, Nikah, 38)
\textbf{İçerik:}
\begin{itemize}
\item Furkan 74:
\end{itemize}
رَبَّنَا هَبْ لَنَا مِنْ أَزْوَاجِنَا وَذُرِّيَّاتِنَا قُرَّةَ أَعْيُنٍ وَاجْعَلْنَا لِلْمُتَّقِينَ إِمَامًا
*Rabbenâ heb lenâ min ezvâcinâ ve zürriyyâtinâ kurrete a'yünin vec'alnâ lil müttekîne imâmâ*
(Rabbimiz! Bize gözümüzü aydınlatacak eşler ve zürriyetler bağışla ve bizi takva sahiplerine önder kıl)
\begin{itemize}
\item Rum 21:
\end{itemize}
وَمِنْ آيَاتِهِ أَنْ خَلَقَ لَكُم مِّنْ أَنفُسِكُمْ أَزْوَاجًا لِّتَسْكُنُوا إِلَيْهَا
*Ve min âyâtihî en haleka leküm min enfüsiküm ezvâcen li teskünû ileyhâ*
(O'nun ayetlerinden biri de sizin için nefislerinizden eşler yaratmasıdır ki, onlara meyledip huzur bulasınız)
6. \textbf{Rızık Muskası}
\textbf{Rivayet:} İbn Abbas'tan (r.a.) rivayet edildiğine göre, rızık genişliği için bu ayetler tavsiye edilmiştir. (Tirmizi, Deavat, 72)
\textbf{İçerik:}
\begin{itemize}
\item Talak 2-3 (Yukarıda geçti)
\item Necm 39:
\end{itemize}
وَأَن لَّيْسَ لِلْإِنسَانِ إِلَّا مَا سَعَىٰ
*Ve en leyse lil insâni illâ mâ seâ*
(İnsan için ancak çalıştığının karşılığı vardır)
7. \textbf{Nazar ve Haset Muskası}
\textbf{Rivayet:} Hz. Peygamber (s.a.v.), nazardan korunmak için bu sureleri tavsiye etmiştir. (Buhari, Tıb, 35)
\textbf{İçerik:}
\begin{itemize}
\item Felak ve Nas Sureleri (Bkz: Bölüm 5.4 ve 5.5)
\item Kalem 51-52:
\end{itemize}
وَإِن يَكَادُ الَّذِينَ كَفَرُوا لَيُزْلِقُونَكَ بِأَبْصَارِهِمْ
*Ve in yekâdüllezîne keferû le yüzlikûneke bi ebsârihim*
(O inkâr edenler Zikr'i (Kur'an'ı) işittikleri zaman, neredeyse seni gözleriyle devireceklerdi)
8. \textbf{Evlilik Muskası}
\textbf{Rivayet:} Hz. Peygamber (s.a.v.), evlilik öncesi ve sonrası için bu ayetleri tavsiye etmiştir. (Ebu Davud, Nikah, 40)
\textbf{İçerik:}
\begin{itemize}
\item Nur 32:
\end{itemize}
وَأَنكِحُوا الْأَيَامَىٰ مِنكُمْ وَالصَّالِحِينَ مِنْ عِبَادِكُمْ وَإِمَائِكُمْ
*Ve enkihül eyâmâ minküm ves sâlihîne min ibâdiküm ve imâiküm*
(İçinizden bekâr olanları, kölelerinizden ve cariyelerinizden uygun olanları evlendirin)
\begin{itemize}
\item Rum 21:
\end{itemize}
وَمِنْ آيَاتِهِ أَنْ خَلَقَ لَكُم مِّنْ أَنفُسِكُمْ أَزْوَاجًا لِّتَسْكُنُوا إِلَيْهَا
*Ve min âyâtihî en haleka leküm min enfüsiküm ezvâcen li teskünû ileyhâ*
(O'nun ayetlerinden biri de sizin için nefislerinizden eşler yaratmasıdır ki, onlara meyledip huzur bulasınız)
\begin{itemize}
\item Rahman 46-47:
\end{itemize}
وَلِمَنْ خَافَ مَقَامَ رَبِّهِ جَنَّتَانِ
*Ve li men hâfe mekâme rabbihî cennetân*
(Rabbinin makamından korkan kimseye iki cennet vardır)
\textbf{Önemli Notlar:}
1. Muska yazımında sadece Kur'an ayetleri ve hadis-i şeriflerde geçen dualar kullanılmalıdır (Buhari, Tıb, 17)
2. Muska içeriğinde anlaşılmayan sembol ve işaretler kullanılmamalıdır (Müslim, Selam, 63)
3. Muskanın yazımı ve taşınması temizlik şartlarına uygun olmalıdır (Ebu Davud, Taharet, 31)
4. Muska suya değmeyecek şekilde muhafaza edilmelidir
5. Muska banyo ve tuvalet gibi yerlere götürülmemelidir
6. Muskanın etkisinin Allah'ın izniyle olduğuna inanılmalıdır
7. Muska, tedavi yerine değil, tedaviye yardımcı olarak kullanılmalıdır (Müslim, Selam, 69)
\textbf{Kullanım Şekli:}
1. Deri, bez veya kağıt üzerine yazılabilir
2. Üç kat muşamba ile kaplanmalıdır
3. Boyunda veya üstte taşınabilir
4. Yatağın başucuna asılabilir
5. Evin giriş kapısının üzerine konulabilir
\textbf{Uyarılar:}
1. Muska, şirk unsurları içermemelidir
2. İçeriği bilinmeyen muskalar kullanılmamalıdır
3. Ticari amaçla muska yazımı doğru değildir
4. Muska, modern tıbbi tedavinin alternatifi değil, tamamlayıcısıdır
\textbf{NOT:} Tüm muska metinlerinin Arapça asılları ve sure-ayet bilgileri kitabın son sayfasında "Muska Metinleri" başlığı altında toplu halde sunulmuştur.
\subsection{3.4. Özel Günler ve Tekrarlar}
\textbf{Bölüm Özeti:}
Bu bölümde, özel günlerde okunması gereken sureler ve tekrarlar ele alınmaktadır. Hz. Peygamber'in (s.a.v.) belirli günlerde uyguladığı özel okuma programları ve bunların faziletleri açıklanmıştır.
*Rivayet:* Özel günlerdeki tekrarlar, Hz. Peygamber'in (s.a.v.) uygulamalarına dayanmaktadır. Tirmizi'de geçen bir hadise göre, Hz. Peygamber (s.a.v.) belirli günlerde özel sureleri tekrarlardı. (Tirmizi, Daavat, 82) İmam Beydavi, "Envarü't-Tenzil" adlı eserinde bu uygulamaların faziletlerini anlatmıştır.
\begin{itemize}
\item Pazartesi: 7 kez İhlas Suresi
\item Çarşamba: 3 kez Ayet-el-Kursi
\item Cuma: 41 kez Felak ve Nas Sureleri
\item Pazar: 33 kez Bakara 285-286
\end{itemize}
\subsection{3.5. Zor Durumlar İçin Özel Uygulamalar}
\textbf{Bölüm Özeti:}
Bu bölümde, zor durumlarda başvurulabilecek özel uygulamalar ve dualar ele alınmaktadır. Hz. Peygamber'in (s.a.v.) ve sahabenin zor durumlarda uyguladığı özel okuma programları, bu durumların üstesinden gelmek için tavsiye edilen sureler ve ayetler detaylı olarak açıklanmıştır.
*Rivayet:* Bu uygulamalar, Hz. Peygamber'in (s.a.v.) ve sahabenin zor durumlarda başvurduğu yöntemlerdir. Müslim'de geçen hadislere göre, Hz. Peygamber (s.a.v.) hastalık ve sıkıntı durumlarında özel sureleri belirli sayılarda okurdu. (Müslim, Zikir, 42) İmam Razi, "Mefatihü'l-Gayb" adlı eserinde bu uygulamaların etkilerini açıklamıştır.
\subsubsection{Büyü ve Nazar İçin}
\textbf{Felak ve Nas Sureleri} (21 gün, günde 7 kez)
Bu sureler tam metin olarak Bölüm 5.4 ve 5.5'te yer almaktadır. Lütfen oradan okuyunuz.
\subsubsection{Hastalık İçin}
\textbf{Şuara 80-81} (40 gün, günde 41 kez)
وَإِذَا مَرِضْتُ فَهُوَ يَشْفِينِ
*Ve izâ meridtü fe hüve yeşfîn*
(Hastalandığım zaman bana şifa veren O'dur)
وَالَّذِي يُمِيتُنِي ثُمَّ يُحْيِينِ
*Vellezîne yümîtünî sümme yuhyîn*
(O, beni öldürecek ve sonra diriltecek olandır)
\subsubsection{Maddi Sıkıntı İçin}
\textbf{İbrahim 7} (21 gün, günde 21 kez)
وَإِذْ تَأَذَّنَ رَبُّكُمْ لَئِن شَكَرْتُمْ لَأَزِيدَنَّكُمْ ۖ وَلَئِن كَفَرْتُمْ إِنَّ عَذَابِي لَشَدِيدٌ
*Ve iz teezzene rabbüküm le in şekertüm le ezîdenneküm, ve le in kefertüm inne azâbî le şedîd*
(Rabbiniz şöyle duyurmuştu: "Andolsun, eğer şükrederseniz elbette size nimetimi artırırım. Eğer nankörlük ederseniz, hiç şüphesiz azabım çok şiddetlidir")
\subsubsection{Ruhsal Huzur İçin}
\textbf{Rad 28} (33 gün, günde 33 kez)
الَّذِينَ آمَنُوا وَتَطْمَئِنُّ قُلُوبُهُم بِذِكْرِ اللَّهِ ۗ أَلَا بِذِكْرِ اللَّهِ تَطْمَئِنُّ الْقُلُوبُ
*Ellezîne âmenû ve tatmainnü kulûbühüm bi zikrillâh. E lâ bi zikrillâhi tatmainnül kulûb*
(Onlar, iman edenler ve kalpleri Allah'ı anmakla huzura kavuşanlardır. Dikkat edin, kalpler ancak Allah'ı anmakla huzur bulur)
\subsection{3.6. Hanefi-Maturidi Mezhebine Göre Özel Uygulamalar}
\textbf{Rivayet:} Bu uygulamalar, Hanefi-Maturidi mezhebinin görüşlerine göre düzenlenmiştir. İmam Maturidi'nin "Kitabü't-Tevhid" adlı eserinde, namaz sonrası ve özel vakitlerdeki zikirlerin önemi vurgulanmıştır. (Buhari, Daavat, 50) İmam Serahsi, "el-Mebsut" adlı eserinde bu uygulamaların fıkhi dayanaklarını açıklamıştır.
\subsubsection{Namaz Sonrası}
\textbf{Ayet-el-Kursi} (1 kez)
Bu ayet tam metin olarak Bölüm 5.2'de yer almaktadır. Lütfen oradan okuyunuz.
\subsubsection{Oruçluyken}
\textbf{İhlas Suresi} (3 kez)
Bu sure tam metin olarak Bölüm 5.3'te yer almaktadır. Lütfen oradan okuyunuz.
\subsubsection{Yatmadan Önce}
\textbf{Felak ve Nas Sureleri} (1'er kez)
Bu sureler tam metin olarak Bölüm 5.4 ve 5.5'te yer almaktadır. Lütfen oradan okuyunuz.
\subsubsection{Uyanınca}
\textbf{Bakara 285-286} (1 kez)
Bu ayetler tam metin olarak Bölüm 6.1'de yer almaktadır. Lütfen oradan okuyunuz.
\subsection{3.7. Günlük Zikir Programı}
\textbf{Rivayet:} Bu program, Hz. Peygamber'in (s.a.v.) günlük zikir uygulamalarına dayanmaktadır. Buhari'de geçen hadislere göre, Hz. Peygamber (s.a.v.) belirli vakitlerde özel zikirleri tekrarlardı. (Buhari, Daavat, 64) İmam Gazali, "İhya-u Ulumid-Din" adlı eserinde bu zikirlerin faziletlerini detaylı olarak anlatmıştır.
\subsubsection{Sabah}
\textbf{Sübhanallahi ve bihamdihi} (100 kez)
سُبْحَانَ اللهِ وَبِحَمْدِهِ
*Sübhanallahi ve bihamdihi*
(Allah'ı tüm eksikliklerden tenzih eder, O'na hamd ederim)
\subsubsection{Öğle}
\textbf{La ilahe illallah} (100 kez)
لَا إِلَهَ إِلَّا اللهُ
*La ilahe illallah*
(Allah'tan başka ilah yoktur)
\subsubsection{İkindi}
\textbf{Allahümme salli ala Muhammed} (100 kez)
اللَّهُمَّ صَلِّ عَلَى مُحَمَّدٍ
*Allahümme salli ala Muhammed*
(Allah'ım, Muhammed'e salat eyle)
\subsubsection{Akşam}
\textbf{Estağfirullah} (100 kez)
أَسْتَغْفِرُ اللهَ
*Estağfirullah*
(Allah'tan bağışlanma dilerim)
\subsubsection{Yatsı}
\textbf{La havle vela kuvvete illa billah} (100 kez)
لَا حَوْلَ وَلَا قُوَّةَ إِلَّا بِاللهِ
*La havle vela kuvvete illa billah*
(Güç ve kuvvet ancak Allah'tandır)

\end{document}
