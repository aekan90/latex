\documentclass[12pt,a4paper]{article}
\usepackage[utf8]{inputenc}
\usepackage[T1]{fontenc}
\usepackage{geometry}

\geometry{margin=2.5cm}
\setlength{\parindent}{0pt}
\setlength{\parskip}{1em}

\begin{document}
\section{2. GİRİŞ}
Bu bölümde, kitabın içeriğine dair genel bilgiler ve önemli notlar yer almaktadır. Ayetlerin ve surelerin doğru kullanımı için temel prensipler açıklanacaktır.
\subsection{2.1. Önemli Notlar}
1. Bu ayetleri okurken niyetinizin halis olması önemlidir
2. Ayetleri okumadan önce abdest almanız önerilir
3. Okuma sırasında konsantre olmanız ve manalarını düşünmeniz tavsiye edilir
4. Düzenli okuma yapmanız önerilir
5. Ayetleri muska olarak taşımak istiyorsanız, ilgili bölümdeki önerileri dikkate alınız
\subsection{2.2. Açıklama}
1. Bu Kur'ân'da seçilmiş ayetler, Diyanet ve Hanefi mezhebi kaynaklarına göre doğrulanmış ayet ve surelerden oluşmaktadır
2. Her ayet için Arapça metin, Türkçe meali ve Latince okunuşu verilmiştir
3. Özel telaffuz gerektiren kelimeler için açıklamalar eklenmiştir
4. Bu muska, korunma ve şifa niyetiyle günlük okunabilir
5. Ancak unutulmamalıdır ki, asıl koruyucu Allah'tır ve bu ayetler O'nun izniyle etki eder
\textbf{Bölüm Özeti:}
Bu bölümde ayetlerin okunması ve uygulanması için gerekli temel bilgiler verilmiş, özellikle niyet, abdest ve düzenli okuma gibi önemli hususlar vurgulanmıştır.
Temel prensipleri ve önemli notları açıkladıktan sonra, şimdi bu bilgilerin günlük hayatta nasıl uygulanacağını gösteren pratik kullanım rehberine geçebiliriz.

\end{document}
