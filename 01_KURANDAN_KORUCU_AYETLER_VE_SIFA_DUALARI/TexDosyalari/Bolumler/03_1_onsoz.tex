\documentclass[12pt,a4paper]{article}
\usepackage[utf8]{inputenc}
\usepackage[T1]{fontenc}
\usepackage{geometry}

\geometry{margin=2.5cm}
\setlength{\parindent}{0pt}
\setlength{\parskip}{1em}

\begin{document}
\section{1. ÖNSÖZ}
Bu bölümde, kitabın temel amacı ve kullanım rehberi ele alınacaktır. Kur'an ve Sünnet'e dayalı bir yaklaşımla hazırlanan bu eser, manevi şifa ve korunma yollarını açıklamaktadır.
*Hadis-i Şerif:* Ey Ümmetim! Size iki emanet bırakıyorum: Allah'ın Kitabı ve benim sünnetim. Bu ikisine sımsıkı sarıldığınız sürece asla sapıtmazsınız. \textit{(Müslim, Hac, 147)}
\textbf{Bölüm Özeti:}
Bu bölümde kitabın temel amacı ve kullanım prensipleri açıklanmıştır. Kur'an ve Sünnet'e bağlılık, doğru anlama ve uygulama, manevi korunma ve bid'atlerden uzak durma konuları vurgulanmıştır.
Kur'an ve Sünnet'in rehberliğinde hazırlanan bu eserin temel prensiplerini açıkladıktan sonra, şimdi kitabın genel yapısı ve kullanımı hakkında önemli bilgileri içeren giriş bölümüne geçebiliriz.

\end{document}
